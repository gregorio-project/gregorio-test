% !TEX TS-program = lualatex
% !TEX encoding = UTF-8

% This is a simple template for a LuaLaTeX document using gregorio scores.

\documentclass[11pt]{article} % use larger type; default would be 10pt

% usual packages loading:
\usepackage{fontspec}
\usepackage{graphicx} % support the \includegraphics command and options
\usepackage{geometry} % See geometry.pdf to learn the layout options. There are lots.
\geometry{a4paper} % or letterpaper (US) or a5paper or....
\usepackage[allowdeprecated=false]{gregoriotex} % for gregorio score inclusion
\usepackage{fullpage} % to reduce the margins
\setmainfont[
    Path = ../../../fonts/ ,
    Extension = .otf ,
    UprightFont = *-Regular ,
    UprightFeatures = { SmallCapsFont = *SC-Regular } ,
    BoldFont = *-Bold ,
    BoldFeatures = { SmallCapsFont = *SC-Bold } ,
    ItalicFont = *-Italic ,
    ItalicFeatures = { SmallCapsFont = *SC-Italic } ,
    BoldItalicFont = *-BoldItalic ,
    BoldItalicFeatures = { SmallCapsFont = *SC-BoldItalic } ,
    Ligatures = TeX
]{Alegreya}

% here we begin the document
\listfiles
\begin{document}

% The title:
\begin{center}\begin{huge}\textsc{Populus Sion}\end{huge}\end{center}

% Here we set the space around the initial.
% Please report to http://home.gna.org/gregorio/gregoriotex/details for more details and options
\grechangedim{afterinitialshift}{2.2mm}{scalable}
\grechangedim{beforeinitialshift}{2.2mm}{scalable}

% Here we set the initial font. Change 43 if you want a bigger initial.
\grechangestyle{initial}{%
\fontsize{43}{43}\selectfont%
}

% We set red lines here, comment it if you want black ones.
\gresetlinecolor{gregoriocolor}

% We set VII above the initial.
\grechangestyle{annotation}{\small\scshape\bfseries}
\greannotation{VII}

% We type a text in the top right corner of the score:
\grechangestyle{commentary}{\small\itshape}
\grecommentary{Cf. Is. 30, 19 . 30 ; Ps. 79}

\grechangenextscorelinedim{2}{spacelinestext}{2cm}{scalable}

% and finally we include the score. The file must be in the same directory as this one.
\gregorioscore[a]{PopulusSion}

\end{document}
